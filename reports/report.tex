\documentclass[11pt,a4paper]{article}
\usepackage[utf8]{inputenc}
\usepackage[francais]{babel}
\usepackage[T1]{fontenc}
\usepackage{amsmath}
\usepackage{amsfonts}
\usepackage{amssymb}
\usepackage{hyperref}

\title{Projet d'option GSI Vivaldi \\ Rapport final}
\author{Nicolas Joseph, Raphaël Gaschignard\\ Guillaume Blondeau, Cyprien Quilici, Jacob Tardieu}

\begin{document}
\maketitle
\section{Le Contexte}

Le projet s'est déroulé est en réalité un sous-projet au projet \texttt{DisCoVEry}. Ce projet qui est né suite à l’impulsion du département de recherche en informatique de l'École des Mines de Nantes et consiste à explorer ce que pourrait être le modèle d’après cloud qui est en pleine expansion. En effet, il apparaît que ce modèle est construit autour de centre de données centralisés qui eux même peuvent être excentrés géographiquement des zones d’utilisation. De cette remarque vient  le modèle \texttt{DisCoVEry} : rapporter les serveurs auprès des utilisateurs. Plus précisément les rapporter au niveau des différents noeuds de réseau, de manière à construire un cloud complètement décentralisé.

\section{La Problematique}
Au sein d'un réseau de type cloud, il faut pouvoir migrer des VM \footnote{Machines Virtuelles} sur des machines differentes grâce à une notion de localité. C'est là ou notre projet intervient. Grâce à un algorythme prédéfinie, Vivaldi, notre brique logicielle doit être capable de fournir les noeuds les plus proches de la machine courante.\\

Cette brique logicielle devait être completement indépendante des autres systèmes \texttt{DisCoVEry} de manière à pouvoir être remplacée facilement le cas échéant.

\section{Méthodologie}
\subsection{Gestion de Projet}
Ce projet a été réalisé en autonomie avec des réunions régulières suivant une méthode de développement agile. Nous reviendront plus en détail sur le détail de la méthodologie de développement dans la partie \ref{subsec:dev}. Nous avons choisi un chef de projet qui avait un rôle de management administratif du projet. Il a géré une partie des relations de l'équipe avec les tuteurs et les éléments de gestion de projet comme le suivi d'avancement des differentes tâches, la répartition des tâches ...\\

Nous avons utilisés plusieurs outils en ligne pour la gestion de projet :
\begin{itemize}
\item \href{https://www.trello.com}{Trello}
\item \href{http://ppulse.f}{Propulse}
\end{itemize}

Nous avons commencé par utiliser Trello pour mettre en place le cahier des charges et les premières briques de base de Vivaldi et mettre en place differentes dates de rendu. Nous avons ensuite basculé sur Propulse, qui nous semblait un outil un peu plus conventionnel et utile que Trello pour la fin du projet.\\

Nous avons aussi régulièrement fait des réunions avec les differents acteurs du projet, à la fois les clients et les tuteurs école.

\subsection{Developpement}
\label{subsec:dev}

\section{Les Résultats}

\section{Le Bilan du projet}

\begin{thebibliography}{9}

\bibitem{vivaldi}
  Frank Dabek, Russ Cox, M. Frans Kaashoek, Robert Morris,
  \emph{Vivaldi : a decentralized network coordinate system}.
  SIGCOMM 2004.

\end{thebibliography}


\end{document}
